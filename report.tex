\documentclass{article}
\usepackage{amsmath}
\makeatletter

\def\v#1{{\fontfamily{cmtt}\selectfont #1}}


\title{Using Clustering for Community Search}
\author{Jianyu Jiang (3030044036)}

\begin{document}

\maketitle
\section{Problem Definition}
In this assignment, you are required to implement and benchmark community search
algorithm using clustering.
The algorithms should first compute user similarity by
vertex similarity and personal page-rank
and should perform a K-means clustering.

\section{Design and Implementation}
The algorithm has three steps. First, we compute the similarity matrix
by vertex similarity or personalized PageRank. Then, we perform
a K-Means algorithm with this similarity matrix. Then, we compare
the K-Means results.

We compare the result with three models: Purity, Entropy
and Normalized mutual information (NMI).

If $W = {w_1, w_2, ..., w_k}$ is the set of clusters and
$C = {c_1, c_2, ..., c_j}$ is the set classes. Then,

\begin{align}
  purity(W, C) = \dfrac{1}{N}\sum\limits_{k}\max_j{w_j \bigcap c_j}
\end{align}.
a perfect clustering has a purity of 1.

For entropy,

\begin{align}
  H(W) = -\sum\limits_kP(w_k)logP(w_k) = -\sum\limits_k\dfrac{|w_k|}{N}log\dfrac{|w_k|}{N}
\end{align}

The minimum of $H(W)$ is 0 if the clustering is random with respect to class
membership. In that case, knowing that a document
is in a particular cluster does not give us any new information
about what its class might be.

For NMI, it is always a value between 0 to 1.


\subsection{Design and Implementation}
We implemented the algorithm using python. We implemented
the personalized PageRank ourselves and use KMeans implementation
from sklearn.


\subsection{Benchmark}
Evaluation of the algorithm is focused of Purity, Entropy and NMI
of using different similarity matrix and size of clusters.

\begin{table}[tbh]
  \center
  \footnotesize
  \begin{tabular}{c|c|c|c|c|c|c}
    \textbf{k} & \textbf{Purity-PR} & \textbf{Purity} & \textbf{Entropy-PR} & \textbf{Entropy} & \textbf{NMI-PR} & \textbf{NMI} \\
    \hline
    2 & 0.5459 & 1 & 0.01233 & 1 & 0.02577 \\
    \hline
    4 & 0.5470 & 1 & 0.0257 & 1 & 0.00488 \\
    \hline
    8 & 0.5481 & 1 & 0.4995 & 1 & 0.00545 \\
    \hline
    16 & 0.5765 & 1 & 1.042 & 1 & 0.01531 \\
  \end{tabular}
  \caption{Evaluation result.}
  \label{result}
\end{table}


\end{document}
