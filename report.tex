\documentclass{article}

\makeatletter

\def\v#1{{\fontfamily{cmtt}\selectfont #1}}


\title{Nearest Neighbor Search}
\author{Jianyu Jiang (3030044036)}

\begin{document}

\maketitle
\section{Problem Definition}
In this assignment, you are required to implement and benchmark nearest
neighbour similarity queries using
the two-step and multi-step similarity search
algorithms. The algorithms should take as input a query
vector q and should compute the nearest neighbour p of q.

\section{Design and Implementation}
The algorithm has three steps. First, we do a Principle Component Analysis (PCA)
on the dataset. Second, we construct a R-Tree with the dataset in the principle
space. At last, we run the two step and multi-step nearest neighbour search.

\subsection{Principle Component Analysis (PCA)}
For a dataset array $X_{nxd}$, we first centre $X$ and get $X'$. Then,
we run a SVD on $X'$, so that $X' = USV^T$. We get the principle component
$G = U_{n*k}S_{k*k}$. We then insert $G$ into the R-Tree. For a query $q$,
we get its corresponding vector $q'$ by $q' = qV_{n*k}$.


\subsection{Nearest Neighbour Search}
To find the nearest neighbour for a query $q'$, we conduct a two-step and a multi-step
nearest neighbour search. For two-step search, we use the $D$ we calculate in
the reduced space and do a range search of [$q' - D$, $q' + D$]. After that, we
get the final result by comparing all data in the range search.

For the multi-step search, we do incremental R-Tree queries until the distance
in reduced space
is larger than the distance in the original space.
For the two-step neighbour search, we use $D$ as the upper bound distance between
the query $q$ and the neighbour $p$ in reduced space. Suppose we find the nearest
neighbour $ |g' - q'| = min(|p' - q'|)$ in reduced space and $y$ is the nearest neighbour
of $q$. Then, we need to prove that $d(g, q) \geq d(y', q')$. \\
\textbf{Proof} If $d(g, q) \geq d(y', q')$ is not fulfilled, then $d(g, q) < d(y', q')$.
We have
$|g - q||V_{dk}| = |g' - q'| < |y' - q'||V_{dk}|$
then $|g' - q'| < min(|y' - q'||V_{dk}|) = |g' - q'||V_{dk}|$, but
$|V_{dk}| \leq 1$. Thus, we have $d(g, q) \geq d(y', q')$.

\subsection{Benchmark}
Evaluation of the algorithm is focused of the time of constructing the R-Tree
and the average query time.
Queries were generated randomly and
these queries are served to the nearest neighbor search system.
The average
query time for 1k queries is recorded. A simple linear scan algorithm is
used for comparison. The result is showed in Table \ref{result}.

\begin{table}[tbh]
  \center
  \footnotesize
  \begin{tabular}{c|c|c|c|c}
    \textbf{k} & \textbf{Indexing} & \textbf{Two-Step} & \textbf{Multi-Step} & \textbf{Linear}\\
    \hline
    5 & 80.6s & 439ms & 639ms & 350ms\\
    \hline
    10 & 125s & 468ms & 695ms & 350ms \\
    \hline
    15 & 165s & 512ms & 726ms & 350ms \\
    \hline
    20 & 167s & 578ms & 748ms & 342ms \\
    \hline
    25 & 193s & 627ms & 287ms & 350ms \\
    \hline
    28 & 206.8s & 642ms & 198ms & 344ms \\
  \end{tabular}
  \caption{Evaluation result.}
  \label{result}
\end{table}

Table~\ref{result} shows that linear search outperforms two-step and multi-step
nearest neighbour search when k is less than 25. This is because the principle
component of $X$ get 99\% information of the original $X$ when k is larger than
25. For two-step nearest
neighbour search, the distance $D$ is not a good estimation of the reduced
dimension, causing a bad performance.

\end{document}
